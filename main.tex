
%-----------------------------------------------------------------------------------------------------------------------------------------------%
%	The MIT License (MIT)
%
%	Copyright (c) 2021 Philip Empl
%
%	Permission is hereby granted, free of charge, to any person obtaining a copy
%	of this software and associated documentation files (the "Software"), to deal
%	in the Software without restriction, including without limitation the rights
%	to use, copy, modify, merge, publish, distribute, sublicense, and/or sell
%	copies of the Software, and to permit persons to whom the Software is
%	furnished to do so, subject to the following conditions:
%	
%	THE SOFTWARE IS PROVIDED "AS IS", WITHOUT WARRANTY OF ANY KIND, EXPRESS OR
%	IMPLIED, INCLUDING BUT NOT LIMITED TO THE WARRANTIES OF MERCHANTABILITY,
%	FITNESS FOR A PARTICULAR PURPOSE AND NONINFRINGEMENT. IN NO EVENT SHALL THE
%	AUTHORS OR COPYRIGHT HOLDERS BE LIABLE FOR ANY CLAIM, DAMAGES OR OTHER
%	LIABILITY, WHETHER IN AN ACTION OF CONTRACT, TORT OR OTHERWISE, ARISING FROM,
%	OUT OF OR IN CONNECTION WITH THE SOFTWARE OR THE USE OR OTHER DEALINGS IN
%	THE SOFTWARE.
%	
%
%-----------------------------------------------------------------------------------------------------------------------------------------------%


%============================================================================%
%
%	DOCUMENT DEFINITION
%
%============================================================================%

\documentclass[10pt,A4,english]{article}	


%----------------------------------------------------------------------------------------
%	ENCODING
%----------------------------------------------------------------------------------------

% we use utf8 since we want to build from any machine
\usepackage[utf8]{inputenc}		
\usepackage[USenglish]{isodate}
\usepackage{fancyhdr}
\usepackage[numbers]{natbib}

%----------------------------------------------------------------------------------------
%	LOGIC
%----------------------------------------------------------------------------------------

% provides \isempty test
\usepackage{xstring, xifthen}
\usepackage{enumitem}
\usepackage[english]{babel}
\usepackage{blindtext}
\usepackage{pdfpages}
\usepackage{changepage}
%----------------------------------------------------------------------------------------
%	FONT BASICS
%----------------------------------------------------------------------------------------

% some tex-live fonts - choose your own

%\usepackage[defaultsans]{droidsans}
%\usepackage[default]{comfortaa}
%\usepackage{cmbright}
\usepackage[default]{raleway}
%\usepackage{fetamont}
%\usepackage[default]{gillius}
%\usepackage[light,math]{iwona}
%\usepackage[thin]{roboto} 

% set font default
\renewcommand*\familydefault{\sfdefault} 	
\usepackage[T1]{fontenc}

% more font size definitions
\usepackage{moresize}

% additional packages
\usepackage{amssymb}
\usepackage{twemojis}
\usepackage{comment}

%----------------------------------------------------------------------------------------
%	FONT AWESOME ICONS
%---------------------------------------------------------------------------------------- 

% include the fontawesome icon set
\usepackage{fontawesome5}

% use to vertically center content
% credits to: http://tex.stackexchange.com/questions/7219/how-to-vertically-center-two-images-next-to-each-other
\newcommand{\vcenteredinclude}[1]{\begingroup
\setbox0=\hbox{\includegraphics{#1}}%
\parbox{\wd0}{\box0}\endgroup}

\newcommand{\tab}[1]{\hspace{.2\textwidth}\rlap{#1}}
% use to vertically center content
% credits to: http://tex.stackexchange.com/questions/7219/how-to-vertically-center-two-images-next-to-each-other
\newcommand*{\vcenteredhbox}[1]{\begingroup
\setbox0=\hbox{#1}\parbox{\wd0}{\box0}\endgroup}

% icon shortcut
\newcommand{\icon}[3]{ 							
	\makebox(#2, #2){\textcolor{maincol}{\csname fa#1\endcsname}}
}	


% icon with text shortcut
\newcommand{\icontext}[4]{ 						
	\vcenteredhbox{\icon{#1}{#2}{#3}}  \hspace{-4pt}  \parbox{0.9\mpwidth}{\textcolor{#4}{#3}}
}

% icon with website url
\newcommand{\iconhref}[5]{ 						
    \vcenteredhbox{\icon{#1}{#2}{#5}}  \hspace{-4pt} \href{#4}{\textcolor{#5}{#3}}
}

% icon with email link
\newcommand{\iconemail}[5]{ 						
    \vcenteredhbox{\icon{#1}{#2}{#5}}  \hspace{-4pt} \href{mailto:#4}{\textcolor{#5}{#3}}
}

%----------------------------------------------------------------------------------------
%	PAGE LAYOUT  DEFINITIONS
%----------------------------------------------------------------------------------------

% page outer frames (debug-only)
% \usepackage{showframe}		

% we use paracol to display breakable two columns
\usepackage{paracol}
\usepackage{tikzpagenodes}
\usetikzlibrary{calc}
\usepackage{lmodern}
\usepackage{multicol}
\usepackage{lipsum}
\usepackage{atbegshi}
% define page styles using geometry
\usepackage[a4paper]{geometry}

% remove all possible margins
\geometry{top=1cm, bottom=1cm, left=1cm, right=1cm}

\usepackage{fancyhdr}
\pagestyle{empty}

% space between header and content
% \setlength{\headheight}{0pt}

% indentation is zero
\setlength{\parindent}{0mm}

%----------------------------------------------------------------------------------------
%	TABLE /ARRAY DEFINITIONS
%---------------------------------------------------------------------------------------- 

% extended aligning of tabular cells
\usepackage{array}

% custom column right-align with fixed width
% use like p{size} but via x{size}
\newcolumntype{x}[1]{%
>{\raggedleft\hspace{0pt}}p{#1}}%


%----------------------------------------------------------------------------------------
%	GRAPHICS DEFINITIONS
%---------------------------------------------------------------------------------------- 

%for header image
\usepackage{graphicx}

% use this for floating figures
% \usepackage{wrapfig}
% \usepackage{float}
% \floatstyle{boxed} 
% \restylefloat{figure}

%for drawing graphics		
\usepackage{tikz}			
\usepackage{ragged2e}	
\usetikzlibrary{shapes, backgrounds,mindmap, trees}

%----------------------------------------------------------------------------------------
%	Color DEFINITIONS
%---------------------------------------------------------------------------------------- 
\usepackage{transparent}
\usepackage{color}

% primary color
\definecolor{maincol}{RGB}{64,64,64}

% accent color, secondary
% \definecolor{accentcol}{RGB}{ 250, 150, 10 }

% dark color
\definecolor{darkcol}{RGB}{70,70,70}

% light color
\definecolor{lightcol}{RGB}{240,240,240} 	% light color
\definecolor{accentcol}{RGB}{13,83,148}		% main theme



% Package for links, must be the last package used
\usepackage[hidelinks]{hyperref}

% returns minipage width minus two times \fboxsep
% to keep padding included in width calculations
% can also be used for other boxes / environments
\newcommand{\mpwidth}{\linewidth-\fboxsep-\fboxsep}
	


%============================================================================%
%
%	CV COMMANDS
%
%============================================================================%

%----------------------------------------------------------------------------------------
%	 CV LIST
%----------------------------------------------------------------------------------------

% renders a standard latex list but abstracts away the environment definition (begin/end)
\newcommand{\cvlist}[1]{
	\begin{itemize}
		#1
	\end{itemize}
}

%----------------------------------------------------------------------------------------
%	 CV TEXT
%----------------------------------------------------------------------------------------

% base class to wrap any text based stuff here. Renders like a paragraph.
% Allows complex commands to be passed, too.
% param 1: *any
\newcommand{\cvtext}[1] {
	\begin{tabular*}{1\mpwidth}{p{0.98\mpwidth}}
		\parbox{1\mpwidth}{#1}
	\end{tabular*}
}
\newcommand{\cvtextsmall}[1] {
	\begin{tabular*}{0.8\mpwidth}{p{0.8\mpwidth}}
		\parbox{0.8\mpwidth}{#1}
	\end{tabular*}
}
%----------------------------------------------------------------------------------------
%	CV SECTION
%----------------------------------------------------------------------------------------

% Renders a a CV section headline with a nice underline in main color.
% param 1: section title
\newcommand{\cvsection}[1] {
	\vspace{14pt}
	\cvtext{
		\textbf{\LARGE{\textcolor{darkcol}{#1}}}\\[-4pt]
		\textcolor{accentcol}{ \rule{0.2\textwidth}{1.5pt} } \\
	}
}

\newcommand{\cvsectionsmall}[1] {
	\vspace{14pt}
	\cvtext{
		\textbf{\Large{\textcolor{darkcol}{#1}}}\\[-4pt]
		\textcolor{accentcol}{ \rule{0.2\textwidth}{1.5pt} } \\
	}
}

\newcommand{\cvheadline}[1] {
	\vspace{16pt}
	\cvtext{
		\textbf{\Huge{\textcolor{accentcol}{#1}}}\\[-4pt]
		 
	}
}

\newcommand{\cvsubheadline}[1] {
	\vspace{16pt}
	\cvtext{
		\textbf{\huge{\textcolor{darkcol}{#1}}}\\[-4pt]
		 
	}
}
%----------------------------------------------------------------------------------------
%	META SKILL
%----------------------------------------------------------------------------------------

% Renders a progress-bar to indicate a certain skill in percent.
% param 1: Icon
% param 2: Icon size
% param 3: name of the skill / tech / etc.
% param 4: percent, values range from 0 to 1
% \cvskill{<Icon>}{<Size>}{<Text>}{<Progress>}
\newcommand{\cvskill}[4] {
	\begin{tabular*}{1\mpwidth}{p{1\mpwidth}}
 		\icontext{#1}{#2}{#3}{black}
	\end{tabular*}%

	\vspace{2pt}
	\hspace{4pt}
	\begin{tikzpicture}[scale=1,rounded corners=2pt,very thin]
		\fill [lightcol!80!black] (0,0) rectangle (1\mpwidth, 0.15);
		\fill [accentcol] (0,0) rectangle (#4\mpwidth, 0.15);
  	\end{tikzpicture}%
}

%----------------------------------------------------------------------------------------
%	LANGUAGES
%----------------------------------------------------------------------------------------

% Renders a progress-bar to indicate a certain skill in percent.
% param 1: country flag.
% param 2: language
% param 3: level (text)
% param 4: percent, values range from 0 to 1
% \cvlanguage{<Flag>}{<Language>}{<Level>}{<Progress>}
\newcommand{\cvlanguage}[4] {
	\begin{tabular*}{1\mpwidth}{p{0.72\mpwidth}  r}
		{\raisebox{-0.10em}{\Large\texttwemoji{flag: #1}}~~\textcolor{black}{\textbf{#2}}} & 
		\textcolor{maincol}{#3}\\
	\end{tabular*}%
	
	\hspace{4pt}
	\begin{tikzpicture}[scale=1,rounded corners=2pt,very thin]
		\fill [lightcol!80!black] (0,0) rectangle (1\mpwidth, 0.15);
		\fill [accentcol] (0,0) rectangle (#4\mpwidth, 0.15);
	\end{tikzpicture}%
}


%----------------------------------------------------------------------------------------
%	 CV EVENT
%----------------------------------------------------------------------------------------

% Renders a table and a paragraph (cvtext) wrapped in a parbox (to ensure minimum content
% is glued together when a pagebreak appears).
% Additional Information can be passed in text or list form (or other environments).
% the work you did
% param 1: time-frame i.e. Sep 14 - Jan 15 (my birthday!) etc.
% param 2:	 event name (job position etc.)
% param 3: Customer, Employer, Industry
% param 4: Short description
% param 5: work done (optional)
% param 6: technologies include (optional)
% param 7: achievements (optional)
\newcommand{\cvevent}[7] {
	
	% we wrap this part in a parbox, so title and description are not separated on a pagebreak
	% if you need more control on page breaks, remove the parbox
	\parbox{\mpwidth}{
		\begin{tabular*}{1\mpwidth}{p{0.66\mpwidth}  r}
	 		\textcolor{black}{\textbf{#2}} & \colorbox{accentcol}{\makebox[0.32\mpwidth]{\textcolor{white}{\textbf{#1}}}} \\
			\textcolor{maincol}{#3} & \\
		\end{tabular*}\\[8pt]
	
		\ifthenelse{\isempty{#4}}{}{
			\cvtext{#4}\\
		}
	}
	{\color{lightcol!95!black}\hrule}
	\vspace{14pt}
}


%----------------------------------------------------------------------------------------
%	 CV ACADEMIC EVENT
%----------------------------------------------------------------------------------------

% Renders a table and a paragraph (cvtext) wrapped in a parbox (to ensure minimum content
% is glued together when a pagebreak appears).
% Additional Information can be passed in text or list form (or other environments).
% the work you did
% param 1: time-frame i.e. Sep 14 - Jan 15 (my birthday!) etc.
% param 2: academic title
% param 3: institution
% param 4: thesis title

% param 5: work done (optional)
% param 6: technologies include (optional)
% param 7: achievements (optional)
\newcommand{\cvacademicevent}[7] {
	
	% we wrap this part in a parbox, so title and description are not separated on a pagebreak
	% if you need more control on page breaks, remove the parbox
	\parbox{\mpwidth}{
		\begin{tabular*}{1\mpwidth}{p{0.66\mpwidth}  r}
			\textcolor{black}{\textbf{#2}} & \colorbox{accentcol}{\makebox[0.32\mpwidth]{\textcolor{white}{\textbf{#1}}}} \\
			\textcolor{maincol}{#3} & \\
		\end{tabular*}\\[4pt]
		
		\ifthenelse{\isempty{#4}}{}{
			\cvtext{\textcolor{accentcol}{Tesis:} #4}\\
		}
		
		\ifthenelse{\isempty{#5}}{}{
			\vspace{2pt}
			\cvtext{\textcolor{accentcol}{Resultados y logros:}}
			\cvtext{#5}
		}
		
	}
	{\color{lightcol!95!black}\hrule}
	\vspace{16pt}
}



%----------------------------------------------------------------------------------------
%	 CV META EVENT
%----------------------------------------------------------------------------------------

% Renders a CV event on the sidebar
% param 1: title
% param 2: subtitle (optional)
% param 3: customer, employer, etc,. (optional)
% param 4: info text (optional)
\newcommand{\cvmetaevent}[4] {
	\textcolor{maincol} { \cvtext{\textbf{\begin{flushleft}#1\end{flushleft}}}}

	\ifthenelse{\isempty{#2}}{}{
	\textcolor{black} {\cvtext{\textbf{#2}} }
	}

	\ifthenelse{\isempty{#3}}{}{
		\cvtext{{ \textcolor{maincol} {#3} }}\\
	}

	\cvtext{#4}\\[14pt]
}

%---------------------------------------------------------------------------------------
%	QR CODE
%----------------------------------------------------------------------------------------

% Renders a qrcode image (centered, relative to the parentwidth)
% param 1: path
% param 2: square size (0-1)
% param 3: caption
\newcommand{\cvqrcode}[3]{%
	\begin{center}
		\includegraphics[width={#2}\mpwidth]{#1}\\[-2pt]
		{#3}
	\end{center}
}



%\cvqrcode{0.5}{Scan to visit my GitHub profile}


% HEADER AND FOOOTER 
%====================================
\newcommand\Header[1]{%
\begin{tikzpicture}[remember picture,overlay]
\fill[accentcol]
  (current page.north west) -- (current page.north east) --
  ([yshift=50pt]current page.north east|-current page text area.north east) --
  ([yshift=50pt,xshift=-3cm]current page.north|-current page text area.north) --
  ([yshift=10pt,xshift=-5cm]current page.north|-current page text area.north) --
  ([yshift=10pt]current page.north west|-current page text area.north west) -- cycle;
\node[font=\sffamily\bfseries\color{white},anchor=west,
  xshift=0.7cm,yshift=-0.32cm] at (current page.north west)
  {\fontsize{12}{12}\selectfont {#1}};
\end{tikzpicture}%
}

\newcommand\Footer[1]{%
\begin{tikzpicture}[remember picture,overlay]
\fill[lightcol!95!black]
  (current page.south east) -- (current page.south west) --
  ([yshift=-80pt]current page.south east|-current page text area.south east) --
  ([yshift=-80pt,xshift=-6cm]current page.south|-current page text area.south) --
  ([xshift=-2.5cm,yshift=-10pt]current page.south|-current page text area.south) --	
  ([yshift=-10pt]current page.south east|-current page text area.south east) -- cycle;
\node[yshift=0.32cm,xshift=9cm] at (current page.south) {\fontsize{10}{10}\selectfont \textbf{\thepage}};
\end{tikzpicture}%
}


% ITEMIZE STYLE
%====================================
\setlist[itemize]{
	label=\textcolor{accentcol}{\large$\bullet$},
	itemsep=0em,
	leftmargin=*
}


%=====================================
%============================================================================%
%
%
%
%	DOCUMENT CONTENT
%
%
%
%============================================================================%
\begin{document}

\columnratio{0.31}
\setlength{\columnsep}{2.2em}
\setlength{\columnseprule}{4pt}
\colseprulecolor{white}


% LEBENSLAUF HIERE
\AtBeginShipoutFirst{\Header{Curriculum Vitae}\Footer{1}}
\AtBeginShipout{\AtBeginShipoutAddToBox{\Header{Curriculum Vitae}\Footer{2}}}

\newpage

\colseprulecolor{lightcol}
\columnratio{0.31}
\setlength{\columnsep}{2.2em}
\setlength{\columnseprule}{4pt}
\begin{paracol}{2}


\begin{leftcolumn}
%---------------------------------------------------------------------------------------
%	META IMAGE
%----------------------------------------------------------------------------------------
\includegraphics[width=\linewidth]{resources/photo.jpg}	%trimming relative to image size


%---------------------------------------------------------------------------------------
%	META SKILLS
%----------------------------------------------------------------------------------------
\fcolorbox{white}{white}{
	\begin{minipage}[c][1.5cm][c]{1\mpwidth}
		\centering
		\Large{\textbf{\textcolor{black}{Eduardo Salazar Hidalgo}}} \\[2pt]
		\normalsize{ \textcolor{maincol} {Doctor en Sistemas Autónomos de Navegación Aérea y Submarina} }
	\end{minipage}
}\\

\icontext{Baby}{15}{15 de Enero de 1994}{black}\\[2pt]
\icontext{Home}{15}{Tlatlauquitepec, Puebla, México}{black}\\[2pt]
\icontext{MapMarked}{15}{Gustavo A. Madero, CDMX, México}{black}\\[2pt]
\icontext{Mobile}{15}{+52 233 131 5104}{black}\\[2pt]
\iconemail{Envelope}{15}{sal.hid.edu@gmail.com}{sal.hid.edu@gmail.com}{black}\\[2pt]
\iconhref{Github}{15}{github.com/sahiedu}{https://www.github.com/sahiedu}{black}\\[2pt]
\iconhref{Orcid}{15}{orcid.org/0009-0000-2257-6455}{https://orcid.org/0009-0000-2257-6455}{black}\\[2pt]


%---------------------------------------------------------------------------------------
%	SKILLS
%----------------------------------------------------------------------------------------

\cvsection{Habilidades}

\cvskill{Robot}{15}{Programación en ROS}{0.7}\\[-0pt]

\cvskill{Plane}{15}{Manejo de controladores de vuelo PX4 y Ardupilot}{0.6}\\[-0pt]

\cvskill{Code}{15}{Programación en C++}{0.65}\\[-0pt]

\cvskill{Python}{15}{Programación en Python}{0.6}\\[-0pt]

\cvskill{Cogs}{15}{Simulación en MATLAB y Simulink}{0.95}\\[-0pt]

\cvskill{Microchip}{15}{Programación de microcontroladores}{0.85}\\[-0pt]

\cvskill{Bolt}{15}{Diseño de PCB's}{0.7}\\[-0pt]

\cvskill{Linux}{15}{Usuario de Linux}{0.9}\\[-0pt]

\cvskill{Wrench}{15}{Diseño Mecánico en SolidWorks y FreeCAD}{0.75}\\[-0pt]

\cvskill{Leaf}{15}{Edición de textos en \LaTeX}{0.95}\\[-0pt]



%---------------------------------------------------------------------------------------
%	LANGUAGES
%----------------------------------------------------------------------------------------
\cvsection{Idiomas}

\cvlanguage{Mexico}{Español}{Nativo}{1} \\[-2pt]

\cvlanguage{United States}{Inglés}{B2}{0.86} \\[-2pt]



%---------------------------------------------------------------------------------------
%	EDUCATION
%----------------------------------------------------------------------------------------
\begin{comment}
	\cvsection{Education}
	
	\cvmetaevent
	{XX/XXXX - XX/XXXX}
	{Management Information Systems (M.Sc.)}
	{University of Lorem}
	{\textit{Field 1 • Field 2} \newline Master's thesis: \glqq title of Master's thesis\grqq.}
	\cvmetaevent
	{XX/XXXX - XX/XXXX}
	{Managment Information Systems (B.Sc.)}
	{University of Lorem}
	{\textit{Field 1 • Field 2} \newline Bachelor's thesis: \glqq title of Master's thesis\grqq.}
	
	\cvmetaevent
	{XX/XXXX - XX/XXXX}
	{A-Level}
	{High School Lorem}
	{\textit{Field 1 • Field 2 • Field 3 • Field 4 • Field 5}.}
	
\end{comment}


%---------------------------------------------------------------------------------------
%	PROJECTS
%----------------------------------------------------------------------------------------

\cvsection{Proyectos}


\cvqrcode{resources/QRBebopMpc}{0.5}{MPC para dron Bebop (ROS)}

\cvqrcode{resources/QRTurtlebotNavigation}{0.5}{Navegación autónoma TurtleBot3}

\cvqrcode{resources/QRVisualOdometry}{0.5}{Odometría visual monocular}

\cvqrcode{resources/QRTurtlebotControl}{0.5}{Seguimiento de waypoints TurtleBot3}

\cvqrcode{resources/QRBebopExamples}{0.5}{Ejemplos básicos Bebop--ROS}


\end{leftcolumn}
\begin{rightcolumn}
%---------------------------------------------------------------------------------------
%	TITLE  HEADER
%----------------------------------------------------------------------------------------


%---------------------------------------------------------------------------------------
%	PROFILE
%----------------------------------------------------------------------------------------
\cvsection{Perfil}
%\vspace{4pt}

\cvtext{
	
Doctor en Sistemas Autónomos de Navegación Aérea y Submarina, Ingeniero Mecatrónico y Maestro en Ciencias de la Electrónica con especialidad en Automatización. Mi formación se centra en el modelado, simulación y control de sistemas dinámicos, con énfasis en vehículos aéreos no tripulados. He participado en proyectos de robótica, control automático y sistemas embebidos, utilizando herramientas como Linux, MATLAB/Simulink, Python, ROS y PX4/ArduPilot. Tengo experiencia en el diseño, implementación y validación de estrategias de control en simulación y prototipos reales. Mi interés principal es desarrollar una carrera académica y de investigación en robótica y sistemas autónomos.

}


%---------------------------------------------------------------------------------------
%	EDUCATION
%----------------------------------------------------------------------------------------
% Renders a table and a paragraph (cvtext) wrapped in a parbox (to ensure minimum content
% is glued together when a pagebreak appears).
% Additional Information can be passed in text or list form (or other environments).
% the work you did
% param 1: time-frame i.e. Sep 14 - Jan 15 etc.
% param 2:	 event name (job position etc.)
% param 3: Customer, Employer, Industry
% param 4: Short description
% param 5: work done (optional)
% param 6: technologies include (optional)
% param 7: achievements (optional)


%\vspace{4pt}
\cvsection{Trayectoria Académica}
%\vspace{4pt}


\cvacademicevent
	{Sep 2021 -- Ago 2025}
	{Doctorado en Sistemas Autónomos de Navegación Aérea y Submarina}
	{Centro de Investigación y de Estudios Avanzados del Instituto Politécnico Nacional (CINVESTAV-IPN)}
	{Diseño de estrategias de control no lineal para mini aeronaves de ala fija}
	{
		\begin{itemize}
			\item Publicación de artículos en revistas internacionales.
			\item Colaboración en las Jornadas de Puertas Abiertas del LANAVEX.
			\item Asesorías académicas a estudiantes.
		\end{itemize}
	}
	\vfill\null


\cvacademicevent
	{Ago 2018 -- Jul 2020}
	{Maestría en Ciencias de la Electrónica, Opción Automatización}
	{Benemérita Universidad Autónoma de Puebla (BUAP)}
	{Control MPC de un cuadricóptero para el seguimiento de trayectorias basado en odometría visual}
	{
		\begin{itemize}
			\item Publicación de artículos en congresos nacionales.
			\item Participación en seminarios académicos.
			\item Visitas de colaboración con la UDLAP y el INAOE.
		\end{itemize}
	}
	\vfill\null


	
\cvacademicevent
	{Ago 2012 -- Ene 2017}
	{Licenciatura en Ingeniería Mecatrónica}
	{Instituto Tecnológico Superior de Zacapoaxtla (ITSZ)}
	{Diseño e implementación de un controlador PI a un motor de DC utilizando herramientas Open Source}
	{
		\begin{itemize}
			\item Estancias internacionales para fortalecimiento del idioma inglés.
			\item Tesis aprobada con mención honorífica.
			\item Docencia de inglés básico (1 semestre).
		\end{itemize}
	}
	\vfill\null

%---------------------------------------------------------------------------------------
%	CURSOS
%----------------------------------------------------------------------------------------
%\vspace{4pt}
\cvsection{Cursos}
%\vspace{4pt}


\cvevent
{Sep 2020 -- Oct 2020}
{Reconstrucción 3D y localización con sensores RGB-D}
{Dr. Fernando Israel Ireta Muñoz
	\newline
	Institut National de Recherche en Informatique et en Automatique (INRIA)
}
{}
\vfill\null


	
%---------------------------------------------------------------------------------------
%	ESTANCIAS
%----------------------------------------------------------------------------------------
\vspace{4pt}
\cvsection{Estancias Académicas}
\vspace{4pt}


\cvevent
{Junio 2015}
{Excellence in Culture and English Language Study (EXCELS)}
{West Virginia State University}
{}
\vfill\null


\cvevent
{Julio 2016}
{SEP Puebla - Brooklyn College English Language and Culture Program}
{City University of New York\newline Brooklyn College}
{}
\vfill\null





%---------------------------------------------------------------------------------------
%	CERTIFICACIONES
%----------------------------------------------------------------------------------------
\vspace{4pt}
\cvsection{Certificaciones}
\vspace{4pt}


\cvevent
{2020 -- 2022}
{Test of English as a Foreign Language}
{TOEFL--ITP\newline583/677 pts.}
{}
\vfill\null



%---------------------------------------------------------------------------------------
%	PUBLICACIONES
%----------------------------------------------------------------------------------------

\cvsection{Publicaciones}

\begin{itemize}[leftmargin=*, itemsep=6pt]
	
	\item[\textcolor{maincol}{\faFile}]
	Salazar, E.; Lozano, R.; Salazar, S.
	Nonlinear Feedback Linearization Control and Region of Attraction Analysis for a Fixed-Wing UAV.
	2025.
	Drones, 9(9), 606.
	DOI: \href{https://doi.org/10.3390/drones9090606}{10.3390/drones9090606}
	
	\item[\textcolor{maincol}{\faFile}]
	Flores, J.; Salazar, S.; González-Hernández, I.; Rosales, Y.; Lozano, R.; Salazar, E.; Nicolas, B.
	Control of Helicopter Using Virtual Swashplate.
	2024.
	Drones, 8(7), 327.
	DOI: \href{https://doi.org/10.3390/drones8070327}{10.3390/drones8070327}
	
	\item[\textcolor{maincol}{\faFile}]
	Salazar-Hidalgo, E.; Castañeda-Camacho, J.; Martínez-Torres, C.; Martínez-Carranza, J.
	Model-based predictive control for trajectory tracking of a quadrotor.
	2020.
	Memorias del Congreso Nacional de Control Automático.
	ISSN: 2594-2492.
	
	\item[\textcolor{maincol}{\faBook}]
	Salazar-Hidalgo, E.; Castañeda-Camacho, J.; Martínez-Torres, C.; Martínez-Carranza, J.
	Seguimiento de trayectorias de un robot móvil diferencial a través del sistema operativo robótico ROS.
	2020.
	19° Congreso Nacional de Mecatrónica, Desarrollos con enfoque mecatrónico, Capítulo 3, pp.~27--38.
	ISBN: 978-607-9394-22-6.
	
	\item[\textcolor{maincol}{\faBook}]
	Vergara-Betancourt, A.; Salazar-Hidalgo, E.; Zapata-Nava, O.
	Obtención de la función de transferencia de un motor de DC mediante el análisis de la curva de reacción.
	2017.
	Revista de Aplicación Científica y Técnica, 3(10), 1--10.
	
	\item[\textcolor{maincol}{\faBook}]
	Vergara-Betancourt, A.; Salazar-Hidalgo, E.; Ramiro-Juárez, J.
	Control de velocidad PI de un motor de DC utilizando herramientas open source.
	2017.
	Revista de Tecnología e Innovación, 4(11), 1--13.
	
\end{itemize}



% hofixes to create fake-space to ensure the whole height is used
\mbox{}
\vfill
\mbox{}
\vfill
\mbox{}
\vfill
\mbox{}
\vfill
\mbox{}
\vfill
\mbox{}
\vfill
\mbox{}
\vfill
\mbox{}



% SIGNATURE
\begin{tabular}{@{}l@{}c@{}p{1cm}@{}p{6.3cm}@{}l@{}}
	& \multicolumn{1}{l}{}            & \multicolumn{1}{l}{} & \multicolumn{1}{l}{}    &  \\
	& Ciudad de México, Diciembre 2025 &  &  \centering{\includegraphics[height=1.5cm]{resources/firma}}  &  \\
 	\cline{4-4}
	&  &                  & \centering Dr. Eduardo Salazar Hidalgo & 
\end{tabular}




\end{rightcolumn}
\end{paracol}


\end{document}


